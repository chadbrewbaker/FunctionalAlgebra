\documentclass{beamer}

% \usepackage{beamerthemesplit} // Activate for custom appearance

\usepackage{listings}
\lstnewenvironment{code}{\lstset{language=Haskell,basicstyle=\small}}{}

\title{Functional Algebra}
\author{Chad Brewbaker}
\date{May 5, 2016}

\begin{document}

\frame{\titlepage}

\section[Outline]{}
\frame{\tableofcontents}

\section{Introduction}
\subsection{Overview of the Beamer Class}

\begin{frame}[fragile]
Sum
$$ A + B $$
\begin{code} 
data X = A | B
f :: X -> Either A B
\end{code}

\end{frame}


 
 




\begin{frame}[fragile]
Product
$$ A \times B$$
\begin{code}
data X = X A B
f :: X -> (A, B)
\end{code}
\end{frame}

\begin{frame}[fragile]
Exponential
$$ B^{A} $$
\begin{code}
f :: A -> B
\end{code}
\end{frame}


\begin{frame}[fragile]
Constants are functions with no arguments
$$a = a^{1} $$
\begin{code}
f ::  A
f' :: () -> A
\end{code}
\end{frame}




\begin{frame}[fragile]
Derivative
$$ {d \over dx}(ax^{n}) = a \times n \times x^{n-1}$$
\begin{code}
data N = M | 1
f ::  (a, N -> x)
f' :: (a, N, M -> X)
\end{code}
\end{frame}

\begin{frame}[fragile]
Curry
$$ (a^{m})^{n} = a^{mn}$$

\begin{code}
curry :: n -> m -> a
curry' :: (n,m) -> a

\end{code}
\end{frame}

\begin{frame}[fragile]
Co-curry
$$ a^{m}b^{m} = (ab)^{m}$$

\begin{code}
f :: (m -> a, m -> b)
f' :: m -> (a,b)
\end{code}
\end{frame}

\begin{frame}[fragile]
Domain Splitting
$$a^{m}a^{n} = a^{m+n}$$
\begin{code}
data B = M | N
f :: (M -> a, N -> a)
f' :: B -> a
\end{code}
\end{frame}

\begin{frame}[fragile]
Function Domain Shrinking
$$ a^{m} \div a^{n} = a^{m-n}$$
\begin{code}
data M = B | N
f :: (M -> a) remove (N -> a)
f' :: B -> a
\end{code}
\end{frame}

\begin{frame}[fragile]
Function Co-domain Shrinking
$$ a^{m} \div b^{m} = ({a \over b})^{m}$$
\begin{code}
data A = N | B
f :: (m -> A) remove (m -> B)
f' :: m -> N
\end{code}
\end{frame}





\frame
{
  \frametitle{Features of the Beamer Class}

  \begin{itemize}
  \item<1-> Normal LaTeX class.
  \item<2-> Easy overlays.
  \item<3-> No external programs needed.      
  \end{itemize}
}
\end{document}
